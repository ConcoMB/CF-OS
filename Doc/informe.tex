\documentclass[10pt,journal,compsoc]{IEEEtran}
\ifCLASSOPTIONcompsoc
\else
\fi
\usepackage{grffile}
\usepackage[pdftex]{graphicx}
\usepackage{float}
\usepackage{perpage}
\MakeSorted{figure}
\MakeSorted{table}
\ifCLASSINFOpdf
\else
\fi
\usepackage[cmex10]{amsmath}
\usepackage{morefloats}
\parskip 7.5pt
\hyphenation{op-tical net-works semi-conduc-tor u-san-do u-sa-mos}

\begin{document}

\title{Sistemas Operativos - Filesystems, IPCs y Servidores Concurrentes - Informe}

\author{Federico Ramundo(51596),~\IEEEmembership{I.T.B.A,}
	Conrado Mader Blanco(51270),~\IEEEmembership{I.T.B.A,}
        Tom\'as Mehdi(51014),~\IEEEmembership{I.T.B.A,}
}

\IEEEcompsoctitleabstractindextext{
\begin{abstract}

En el presente informe se describiran cuestiones de dise�o, implementaci\'o y problemas a lo largo de 
la implementacion del TPE1 de Sistemas Operativos. El trabajo se realiz\'o en lenguaje c.

\end{abstract}}
\maketitle

\IEEEdisplaynotcompsoctitleabstractindextext

\IEEEpeerreviewmaketitle


\section{Objetivo}

\IEEEPARstart{E}{}l objetivo de este trabajo es familiarizarse con el uso de sistemas 
cliente-servidor concurrentes, implementando el servidor mediante la creaci\'on de procesos
hijos utilizando fork() y mediante la creaci\'on de threads. Al mismo tiempo,
ejercitar el uso de los distintos tipos de primitivas de sincronizaci\'on y
comunicaci\'on de procesos (IPC) y manejar con autoridad el filesystem de Linuz desde
el lado usuario. En la segunda secci\'on se tratara la implementaci\'on. En la 
tercera los problemas y cambios a medida que se avanzaba. En la cuarta y ultima secci\'on se
discutiran si los objetivos fueron alcanzados y de que manera.

\section{Implementaci\'on y problemas}
El trabajo practico se realico en ansi C. Se codific\'o todo para FIFOs o pipes nombrados y luego se
modulariz\'o para el resto de las comunicaciones. 
\subsection{Estructuras}
Se crearon las siguientes estructuras de datos:
\begin{itemize}
 \item user\_t
 \item client\_t
 \item team\_t
 \item sportist\_t
 \item draft\_t
 \item league\_t
 \item trade\_t
\end{itemize}
\subsection{Problemas}
\section{Conclusiones}

\begin{thebibliography}{1}

\bibitem{Libro}
W. Richard Stevens, Addison Wesley, \emph{Advanced Programming in the UNIX Environment}
\bibitem{Clases}
Etchegoyen, Hugo Eduardo, \emph{Archivo word}
\bibitem{Internet}
Searchs on google
\end{thebibliography}

\end{document}